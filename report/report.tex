\documentclass[12pt]{article}
\usepackage{a4wide}
\usepackage{latexsym}
\usepackage{amssymb}
\usepackage{epic}
\usepackage{graphicx}
%\pagestyle{empty}
\newcommand{\tr}{\mbox{\sf true}}
\newcommand{\fa}{\mbox{\sf false}}
\newcommand{\bimp}{\leftrightarrow}

\begin{document}
\section*{Automated Reasoning\\Assignment 1}

\begin{center}
Wouter Geraedts \\
Judith van Stegeren\\
\end{center}

\vspace{8mm}

\subsection*{Trucks and pallets}
The problem: find a way to load pallets of different objects onto 6 trucks, while adhering to various constraints. Investigate what is the maximum number of pallets of prittles that can be delivered, and show how for that number all pallets may be divided over the six trucks.

We have generalized the problem in the following way: given $t$ trucks with a capacity of $c$ that can carry at most $p$ pallets, a list of numbers of pallets of nuzzles, skipples, crottles and dupples and a list of the weight of a pallet of each of these items, maximize the number of pallets of prittles that can be delivered.

We start out by introducing our representation. We have a function 
\[\texttt{Pallets :: Int -> Int -> Int}\]
\texttt{Pallets t o} returns the number of pallets of object \texttt{o} that truck \texttt{t} is carrying.

Trucks cannot carry a negative amount of pallets:
\[ \bigwedge_{0 \le i < t} ( \bigwedge_{0 \le j < o} \texttt{Pallets i j} \ge 0 ) \]

Trucks cannot carry more than $p$ pallets:
\[ \bigwedge_{0 \le i < t} \texttt{(Pallets i 0)} + \texttt{(Pallets i 1)} \dots + \texttt{(Pallets i o)} \le p\]

Each pallet of object \texttt{o} weighs \texttt{weight[o]} kg. Trucks cannot carry more than the maximum weight of $c$ kg:
\[ \bigwedge_{0 \le i < t} \texttt{(Pallets i 0)} \cdot \texttt{weight[0]} + \dots + \texttt{(Pallets i o)} \cdot \texttt{weight[o]} \le c \]
Prittles and crottles are not allowed in the same truck. In other words, for every truck \texttt{t} it should not be possible that \texttt{(Pallets t prittles)} and \texttt{(Pallets t crottles)} are both greater than 0:
\[ \bigwedge_{0 \le i < t} \neg ( \texttt{Pallets i prittles} > 0) \wedge (\texttt{Pallets i crottles} > 0 ))\]

//hier ben ik niet helemaal tevreden over:
There can be at most 2 trucks carrying skipples. To be able to express this constraint, we first need to count the number of trucks carrying skipples. We introduce a variable \texttt{skt} to count the number of skipple-carrying trucks. We calculate the value of this integer by summing over all trucks \texttt{t} with a simple if-then-else-statement in yices: 
\[ \texttt{skt} = \sum_{0\le i < t} \texttt{(ite (> (Pallets i skipples) 0) 1 0)}\]

Now we can easily express the constraint by adding the following to our formula:
\[ \texttt{skt} \le 2 \]


\subsection*{Computation steps}
The problem: given a set of $N$ variables $a_1, \cdots, a_N$ and their initial value.
For all variables except $a_1$ and $a_N$, it is possible to either do nothing, or assign a new value to the variable:
\[ a_v = a_{v-1} + a_{v+1} \]
We need to find the minimum number of steps required for any of these variables to attain the value 157.
We compute this by fixing the number of steps $T$, and then defining a function
\[\texttt{A :: Int -> Int -> Int}\]
such that \texttt{A v t} is the value of variable $a_v$ after $t$ steps.

First we define the initial value for all variables:
\[ \bigwedge_{1 \le v \le N} \texttt{A v 0} = v \]
Then we define the steps allowed for each variable.
For the first and last variable we have:
\[ \bigwedge_{v \in \{1, N\}} \bigwedge_{0 \le t < T} \texttt{A v (t+1)} = \texttt{A v t} \]
For all other variables we have:
\[ \bigwedge_{1 < v < N} \bigwedge_{0 \le t < T} (\texttt{A v (t+1)} = \texttt{A v t}) \vee (\texttt{A v (t+1)} = \texttt{A (v-1) t} + \texttt{A (v+1) t}) \]
Finally we define the end-goal:
\[ \bigvee_{1 \le v \le N} \texttt{A v T} = 157 \]

To find the minimum number of steps required, we use binary search on the number of steps $T$.
If for any $T$ the problem is satisfiable, the problem is also satisfiable for $T+1$, because the ``do nothing''-step can always be chosen.
Thus binary search is applicable in this case.

Our algorithm for binary search concludes that the problem can be solved in at least 6 steps.

\end{document}
